\documentclass[a5paper,12pt]{article}
\usepackage[utf8]{inputenc}
\usepackage[T1]{fontenc}
\usepackage[greek,french]{babel} 
\usepackage{fancyhdr}
\usepackage{amsthm}
\usepackage{amsmath}
\usepackage{parskip}
\usepackage{tikz}
\usetikzlibrary{patterns}
\usepackage{tkz-euclide}
\usepackage{mdframed}
\usepackage{xcolor}
\usepackage[margin=1cm,top=1.3cm]{geometry}
\usepackage{lastpage}
\usetikzlibrary{shapes.misc}

\tikzset{cross/.style={cross out, draw, 
         minimum size=2*(#1-\pgflinewidth), 
         inner sep=0pt, outer sep=0pt},
         cross/.default={2pt}}

\newmdenv[linecolor=black,innertopmargin=1em,innerbottommargin=1em,backgroundcolor=lightgray!50!white]{infobox}
\newmdenv[linecolor=black,innertopmargin=1em,innerbottommargin=1em,backgroundcolor=white]{whitebox}

\setlength\headheight{24pt} 

\theoremstyle{definition}
\newtheorem*{dfns}{Définitions}
\newtheorem*{dfn}{Définition}
\newtheorem*{prop}{Propriétés}


\begin{document}
\pagestyle{fancy}
%... then configure it.
\fancyhead{} % clear all header fields
\fancyhead[L]{Leçon}
\fancyhead[C]{Multiples et Diviseurs}
\fancyhead[R]{\today}
\fancyfoot{} % clear all footer fields

\begin{infobox}
 On dit qu'un nombre est \textbf{divisible} par un autre s'il n'y a pas de reste quand on fait la division euclidienne.
 
 \textbf{Examples :}
 \begin{itemize}\item 18 est divisible par 2 car $18 = 2\times 9+0$. \item 16 n'est pas divisible par 5 car $16=5\times3+1$.
 \item 125 \dotfill\ par 7 car $125=7\times....+...$
 \item 534 \dotfill\ par 3 car $534=3\times....+...$  
 \end{itemize}
\end{infobox}
\begin{whitebox}
 \textbf{Toutes ces phrases signifient la même chose :}\begin{itemize}
 \item $A$ est divisible par $B$.
 \item On peut diviser $A$ par $B$.
 \item $B$ est un diviseur de $A$.
 \item $A$ est un multiple de $B$.
 \item $A$ est dans la table de multiplication de $B$.
 \end{itemize}
\end{whitebox}
Attention à ne pas confondre : $12$ est un multiple de $4$, $4$ est un diviseur de $12$, mais $4$ n'est pas un multiple de $12$ !

\ 

\begin{infobox}
 \textbf{Critères de divisibilité.} Un nombre est :
 \begin{itemize}
  \item divisible par 2 si son chiffre des unités est 0, 2, 4, 6, ou 8.
  \item divisible par 4 si ses deux derniers chiffres sont soit
  \begin{itemize}
  \item un nombre pair puis 0, 4, ou 8 ;
  \item ou bien un nombre impair puis 2 ou 6.
  \end{itemize}
  \item divisible par 5 si son chiffre des unités est 0 ou 5.
  \item divisible par 10 si son chiffre des unités est 0.
  \item divisible par 3 si la somme de ses chiffres est divisible par 3.
  \item divisible par 9 si la somme de ses chiffres est divisible par 9.
  \item divisible par 6 s'il est divisible par 2 et par 3.
 \end{itemize}
\end{infobox}

Est-ce que 6651416 est divisible par 2 ? par 3 ? par 4 ? par 5 ? par 6 ? par 9 ? par 10 ?

Pour 7 et pour 8 il existe des critères mais ils sont un peu compliqués.

\textbf{Exercices :} Lister tous les diviseurs de 60.

\end{document}
