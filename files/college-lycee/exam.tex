\documentclass[a4paper,12pt]{exam}
%Documentation de la classe exam : https://mirrors.ircam.fr/pub/CTAN/macros/latex/contrib/exam/examdoc.pdf
\usepackage[utf8]{inputenc}
\usepackage[T1]{fontenc}
\usepackage[french]{babel} 
\usepackage{amsmath,amssymb}
\usepackage{titlesec}
\usepackage{tikz}
\usepackage{lastpage}
\usetikzlibrary{shapes.misc}

\tikzset{cross/.style={cross out, draw, 
         minimum size=2*(#1-\pgflinewidth), 
         inner sep=0pt, outer sep=0pt},
         cross/.default={2pt}}
\DeclareMathOperator{\cm}{cm}


%En-tête et pied de page
\pagestyle{headandfoot}
\header{Classe :\\Nom \& Prénom :}{\Large\textbf{Évaluation}}{Page \thepage/\pageref{LastPage}\\\today}
\headrule
\footrule
\footer{}{}{}

\titleformat{\section}[frame]{\Huge\bfseries\filright}{}{2mm}{\centering  }

\printanswers					%Commenter cette ligne pour cacher les solutions.

\qformat{\textbf{Exercice \thequestion~:}\hfill}
\newcommand{\changesolution}[1]{\renewcommand{\solutiontitle}{\noindent\textbf{#1~:}}
}
\changesolution{Figures}

\extrawidth{1cm}

\begin{document}

\section{Triangles égaux}
Entourez la ou les bonnes réponses. Vous pouvez tracer les figures pour vous aider.

\setlength{\tabcolsep}{15pt}
\renewcommand{\arraystretch}{1.5}
\begin{center}
\begin{tabular}{|p{4cm}||p{3cm}|p{3cm}|p{3cm}|}\hline
 On passe du quadrilatère $ABGK$ au quadrilatère $CDIM$ en faisant & une translation & une rotation & ni l'une ni l'autre\\\hline
 On passe du triangle $KGR$ au triangle $HDO$ en faisant & une translation & une rotation & ni l'une ni l'autre \\\hline
 Quels sont les triangles égaux ? & $UVQ$ \& $WRS$ & $ABL$ \& $GND$ & $KGR$ \& $CIN$ \\\hline
 Deux triangles sont égaux si : & leurs côtés sont deux à deux de même longueur & leurs angles sont deux à deux de même mesure & ils ont un côté de même longueur et un angle de même mesure\\\hline 
 Dans les triangles $UPL$ et $VRM$ : & $[UP]$ est homologue à $[MR]$ & $R$ est homologue à $P$ & $U$ est homologue à $V$\\\hline
\end{tabular}

\ 

\ 
\vfill
\begin{tikzpicture}
   \foreach \x in {0,1,2,3,4}{
  \foreach \y in {0,1,2,3,4}{ 
   \node[draw,circle,inner sep=1pt,fill] at (1.5*\x,1.5*\y) {};}}
   \newcounter{mycount}
\setcounter{mycount}{`A}
\foreach \y in {4,3,2,1,0}
  \foreach \x in {0,1,2,3,4}
    \node at (1.5*\x-0.3,1.5*\y+0.2) {$\char\value{mycount}$\addtocounter{mycount}{1}};
  \end{tikzpicture}
\end{center}
\vfill
\end{document}
