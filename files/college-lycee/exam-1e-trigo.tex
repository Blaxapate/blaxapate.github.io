\documentclass[a4paper,12pt]{exam}
%Documentation de la classe exam : https://mirrors.ircam.fr/pub/CTAN/macros/latex/contrib/exam/examdoc.pdf
\usepackage[utf8]{inputenc}
\usepackage[T1]{fontenc}
\usepackage[french]{babel} 
\usepackage{amsmath,amssymb}
\usepackage{titlesec}
\usepackage{pgf,tikz}
\usetikzlibrary{arrows}
\usepackage{lastpage}
\usepackage{pgfplots}
\usepackage{enumerate}
\usepackage[inline]{enumitem}

\DeclareMathOperator{\cm}{cm}
\DeclareMathOperator{\m}{m}


%En-tête et pied de page
\pagestyle{headandfoot}
\header{1\textsuperscript{ère} maths spé}{\Large\textbf{Évaluation} -- Trigonométrie}{
page \thepage/2
%\today
%LLG 2023-24
}
\headrule
\footer{}{}{}

\titleformat{\section}[frame]{\Huge\bfseries\filright}{}{2mm}{\centering  }

%\printanswers					%Commenter cette ligne pour cacher les solutions.


\newcommand{\changesolution}[1]{\renewcommand{\solutiontitle}{\noindent\textbf{#1~:}}
}
\changesolution{Corrigé}

%command \exercice to be called later
\newcommand{\exercice}[2]{\qformat{\textbf{Exercice \thequestion}#1\hfill\textbf{#2}}\question\ }
%command \exo for a blank exercise
\newcommand{\exo}{\exercice{}{\ }}
%command \exop{x} for an exercise worth x points
\newcommand{\exop}[1]{\exercice{}{#1pt}}
%command \exon{x} for an exercise named x
\newcommand{\exon}[1]{\exercice{\textbf{ :}\emph{ #1.}}{\ }}
%commmand \exonp{x,y} for an exercise named x and worth y points
\newcommand{\exonp}[2]{\exercice{\textbf{ :}\emph{ #1.}}{#2pt}}

\extrawidth{1cm}

\begin{document}
\begin{questions}
\exonp{tiré d'un sujet de bac 2016}{10}
\ 
\begin{minipage}{0.48\textwidth}
Lors d'un match de rugby, un joueur doit transformer un essai qui a été marqué au point $E$ (voir figure ci-contre) situé à l'extérieur du segment $[AB]$.

La transformation consiste à taper le ballon par un coup de pied depuis un point $T$ que le joueur a le droit de choisir n'importe où sur le segment $[EM]$ perpendiculaire à la droite $(AB)$ sauf en $E$.

\end{minipage}\begin{minipage}{0.48\textwidth}\begin{center}\includegraphics[scale=0.8]{terrain.png}
\end{center}\end{minipage}
 La transformation est réussie si le ballon passe entre les poteaux repérés par les points $A$ et $B$ sur la figure.


Pour maximiser ses chances de réussite, le joueur tente de déterminer la position du point $T$ qui rend l'angle $\widehat{ATB}$ le plus grand possible.

Le but de cet exercice est donc de rechercher s'il existe une position du point $T$ sur le segment $[EM]$ pour laquelle l'angle $\widehat{ATB}$ est maximum et, si c'est le cas, de déterminer une valeur approchée de cet angle.

Dans toute la suite, on note $x$ la longueur $ET$, qu'on cherche à déterminer.
Les dimensions du terrain sont les suivantes : $EM=50\m$, $EA=25\m$ et $AB=5{,}6\m$. On note $\alpha$ la mesure en radian de l'angle $\widehat{ETA}$, $\beta$ la mesure en radian de l'angle $\widehat{ETB}$ et $\gamma$ la mesure en radian de l'angle $\widehat{ATB}$.

\begin{enumerate}
\item En utilisant les triangles rectangles $ETA$ et $ETB$ ainsi que les longueurs fournies, exprimer $\tan(\alpha)$ et $\tan(\beta)$ en fonction de $x$.
\item Montrer que la fonction $\tan$ est strictement croissante sur l'intervalle $]0\;;\tfrac\pi2[$. La fonction $\tan$ est-elle paire ou impaire ?
\item L'angle $\widehat{ATB}$ admet une mesure $\gamma$ appartenant à l'intervalle $]0\;;\tfrac\pi2[$, résultat admis ici, que l'on peut observer sur la figure.
Montrer que pour tout $a,b\in\mathbb R$ :
$$\tan(a-b)=\frac{\tan(a)-\tan(b)}{1+\tan(a)\times\tan(b)}$$
\item En déduire : $$\tan(\gamma)=\frac{5{,}6x}{x^2+765}$$
\item L'angle $\widehat{ATB}$ est maximum lorsque sa mesure $\gamma$ est maximale.

Montrer que cela correspond à un minimum sur l'intervalle $]0\;;50]$ de la fonction $f$ définie par : $f(x)=x+\tfrac{765}x$. Montrer qu'il existe une unique valeur de $x$ pour laquelle l'angle $\widehat{ATB}$ est maximum et déterminer cette valeur de $x$ au mètre près ainsi qu'une mesure de l'angle $\widehat{ATB}$ à $0{,}01$ radian près. 
\end{enumerate}
\newpage
\exonp{Chaînes de Steiner}{10}
\definecolor{xdxdff}{rgb}{0.66,0.66,0.66}
\definecolor{uququq}{rgb}{0.25,0.25,0.25}
\definecolor{qqqqff}{rgb}{0.33,0.33,0.33}

\begin{minipage}{0.47\textwidth}
Une chaîne de Steiner est composée de deux cercles concentriques, appelés cercles intérieur et extérieur, et d'une chaîne de $n$ cercles de même rayon, tangents au cercle intérieur, au cercle extérieur, et à chacun de leurs deux voisins.

 Sur la figure ci-contre, on a représenté une chaîne de Steiner avec $n=8$. Le point $O$ est le centre des deux cerlces concentrique. On note $R$ le rayon du cercle extérieur, $r$ le rayon du cercle intérieur, et $\rho$ le rayon de chacun des cercles de la chaîne.
 
 Les points $A$ et $B$ sont les centres de deux cercles consécutifs de la chaîne, $H$ est le milieu du segment $[AB]$, et $\theta$ est la mesure de l'angle $\widehat{AOH}$.
\end{minipage}\hfill\begin{minipage}{0.45\textwidth}
\begin{center}
\begin{tikzpicture}[line cap=round,line join=round,>=triangle 45,x=1.0cm,y=1.0cm]
\draw(6.92,10.23) circle (1.1cm);
\draw(8.86,9.19) circle (1.1cm);
\draw(9.48,7.08) circle (1.1cm);
\draw(8.44,5.15) circle (1.1cm);
\draw(6.33,4.52) circle (1.1cm);
\draw(4.4,5.57) circle (1.1cm);
\draw(3.77,7.67) circle (1.1cm);
\draw(4.82,9.61) circle (1.1cm);
\draw(6.63,7.38) circle (3.97cm);
\draw(6.63,7.38) circle (1.77cm);
\draw (3.77,7.67)-- (4.08,6.62);
\draw (4.08,6.62)-- (4.4,5.57);
\draw (4.4,5.57)-- (6.63,7.38);
\draw (4.08,6.62)-- (6.63,7.38);
\draw (3.77,7.67)-- (6.63,7.38);
\draw (6.63,7.38)-- (7.36,5.76);
\draw (6.63,7.38)-- (10.58,7.04);
\begin{scriptsize}
\draw (7,6.8) node[anchor=north west] {$r$};
\draw (8,7.6) node[anchor=north west] {$R$};
\draw (4.82,9.61)-- (5.87,9.29);
\draw (5.3,9.78) node[anchor=north west] {$\rho$};
\fill [color=qqqqff] (3.77,7.67) circle (1.5pt);
\draw[color=qqqqff] (3.9,7.9) node {$A$};
\fill [color=qqqqff] (4.4,5.57) circle (1.5pt);
\draw[color=qqqqff] (4.49,5.4) node {$B$};
\fill [color=uququq] (4.82,9.61) circle (1.5pt);
\draw[color=uququq] (4.9,9.8) node {$C$};
\fill [color=uququq] (4.08,6.62) circle (1.5pt);
\draw[color=uququq] (4.2,6.87) node {$H$};
\fill [color=uququq] (6.63,7.38) circle (1.5pt);
\draw[color=uququq] (6.71,7.6) node {$O$};
\draw[color=uququq] (5.9,7.3) node {$\theta$};
\end{scriptsize}
\draw [shift={(6.63,7.38)}] (0,0) -- (174.06:0.6) arc (174.06:196.56:0.6) -- cycle;
\end{tikzpicture}
\end{center}
\end{minipage}

\begin{enumerate}
 \item Justifier que $R=r+2\rho$ et que $\theta=\frac\pi n$.
 \item Montrer que $\sin(\theta)=\frac\rho{r+\rho}$. En déduire $\rho=\frac{r\sin(\theta)}{1-\sin(\theta)}$.
 \item Montrer finalement la formule de Steiner :
 
 $$\frac Rr=\frac{1+\sin(\theta)}{1-\sin(\theta)}$$
 \item On trace deux cercles concentriques de rayon $r=1$ et $R=3\sqrt{2}$. Peut-on construire une chaîne de Steiner entre ces deux cercles ? Avec combien de cercles ?
 \item À l'aide de la calculatrice, donner une valeur approchée de $\sin(\tfrac{\pi}{20})$ à $10^{-2}$ près. Pour tracer une chaîne de Steiner avec $n=20$ cercles et avec $R=1$, quel sera le rayon $r$ du cercle intérieur ?
\end{enumerate}

\end{questions}
\end{document}
