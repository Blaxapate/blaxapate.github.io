\documentclass[a4paper,landscape,12pt]{article}
\usepackage[utf8]{inputenc}
\usepackage[T1]{fontenc}
\usepackage[greek,french]{babel} 
\usepackage{fancyhdr}
\usepackage{amsthm}
\usepackage{amsmath}
\usepackage{parskip}
\usepackage{tikz}
\usetikzlibrary{patterns}
\usepackage{tkz-euclide}
\usepackage{mdframed}
\usepackage{xcolor}
\usepackage[margin=2cm]{geometry}
\usepackage{lastpage}
\usetikzlibrary{shapes.misc}

\tikzset{cross/.style={cross out, draw, 
         minimum size=2*(#1-\pgflinewidth), 
         inner sep=0pt, outer sep=0pt},
         cross/.default={2pt}}

\newmdenv[linecolor=black,innertopmargin=2em,innerbottommargin=2em,backgroundcolor=lightgray!50!white]{infobox}
\newmdenv[linecolor=black,innertopmargin=2em,innerbottommargin=2em,backgroundcolor=white]{whitebox}

\setlength\headheight{24pt} 

\theoremstyle{definition}
\newtheorem*{dfns}{Définitions}
\newtheorem*{dfn}{Définition}
\newtheorem*{prop}{Propriétés}

\begin{document}
\pagestyle{fancy}
%... then configure it.
\fancyhead{} % clear all header fields
\fancyhead[L]{Isométries}
\fancyhead[C]{\textbf{Leçon à compléter}}
\fancyhead[R]{page \thepage/\pageref{LastPage}}
\fancyfoot{} % clear all footer fields

{\small
\textbf{Isométrie}. Du grec ancien \selectlanguage{greek}ἰσομετρία\selectlanguage{french} (isometría), « même mesure ».

(Géométrie) Transformation qui conserve les distances.
    \emph{Une translation ou une rotation sont des isométries.} -- (src: site internet assistance scolaire)
    
(Littérature) Dans la poésie, utilisation du même mètre dans un poème ou une séquence de vers à l'intérieur d'un poème.
        \emph{Il faut juste oublier de faire de l'isométrie une des bases fondamentale de la métrique} -- (src: Abdallah Bounfour, Introduction à la littérature berbère: La poésie, page 173)
}

Il y a trois types d'isométries : les translations, les rotations, et les symmétries axiales.

\begin{minipage}{0.32\textwidth}
\section*{Translations}

\begin{infobox}
 Une \textbf{translation} consiste à déplacer une figure en la faisant ``glisser'' sans la faire tourner, sans l'agrandir ou la rétrécir, et sans la déformer.
 
 On la caractérise par une direction, un sens, et une distance.
\end{infobox}
\begin{whitebox}
La figure $A'B'C'D'E'F'$ est obtenue par translation de la figure $ABCDEF$ horizontalement, vers la droite, de 4cm. 
 \begin{center}
  \begin{tikzpicture}
   \draw (0,0)node[anchor=north]{$A$} -- (0,2)node[anchor=east]{$B$} -- (1,3)node[anchor=south]{$C$} -- (1,1)node[anchor=east]{$D$} -- (2,2)node[anchor=south]{$E$} -- (2,1)node[anchor=west]{$F$} --cycle;
   \draw (4,0)node[anchor=north]{$A'$} -- (4,2)node[anchor=east]{$B'$} -- (5,3)node[anchor=south]{$C'$} -- (5,1)node[anchor=east]{$D'$} -- (6,2)node[anchor=south]{$E'$} -- (6,1)node[anchor=west]{$F'$} --cycle;
  \end{tikzpicture}
 \end{center}
\end{whitebox}
\vfill
\end{minipage}\hfill\begin{minipage}{0.32\textwidth}

\section*{Rotations}
\begin{infobox}
Une \textbf{rotation} consiste à ``faire tourner'' une figure, sans l'agrandir ou la rétrécir, et sans la déformer.

On la caractérise par un centre de rotation, un angle, et un sens de rotation.
\end{infobox}
\begin{whitebox}
La figure $G'H'I'$ est obtenue par rotation de la figure $GHI$ autour du point $O$, de $90^\circ$, dans le sens des aiguilles d'une montre. 
 \begin{center}
  \begin{tikzpicture}
   \draw (0,0)node(o)[anchor=north east]{$O$};
   \draw (0,0)node[cross]{};
   \draw (1,1)node[anchor=north](g){$G'$} -- (-1,2)node[anchor=east]{$H'$} -- (1,2)node[anchor=south]{$I'$} --cycle;
   \draw (-1,1)node[anchor=west](g'){$G$} -- (-2,-1)node[anchor=north]{$H$} -- (-2,1)node[anchor=south]{$I$} -- cycle;
  \end{tikzpicture}
 \end{center}
\end{whitebox}
\vfill
\end{minipage}\hfill\begin{minipage}{0.32\textwidth}

\section*{Symmétries axiales}

\begin{infobox}
Une \textbf{symmétrie axiale} consiste à ``retourner'' une figure comme dans un miroir, sans l'agrandir ou la rétrécir, et sans la déformer.

On la caractérise par son axe de symmétrie.
\end{infobox}

\begin{whitebox}
La figure $J'K'L'M'$ est obtenue par symmétrie de la figure $JKLM$ par rapport à la droite $(XY)$. 
 \begin{center}
\begin{tikzpicture}
\draw (-2,-1)--(-1,-0.5)node[anchor=north]{$\phantom{M}X$}--(1.8,0.9)node[anchor=south]{$Y$}--(2,1);
\draw (-1.05,-0.4)--(-0.95,-0.6);
\draw (1.75,1)--(1.85,0.8);
\draw (0,1)node[anchor=north]{$\phantom{I}J$} -- (-0.5,0)node[anchor=east]{$K$}--(-0.5,1)node[anchor=south]{$L$}--(1,1.2)node[anchor=south]{$M$}--cycle;
\draw (0.8,-0.6)--(-0.3,-0.4)--(0.5,-1)--(1.56,0.08)--cycle;
\end{tikzpicture}
 \end{center}
\end{whitebox}
\vfill
\end{minipage}

              
\end{document}
