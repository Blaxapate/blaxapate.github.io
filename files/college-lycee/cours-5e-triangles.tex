\documentclass[a4paper,12pt]{article}
\usepackage[utf8]{inputenc}
\usepackage[T1]{fontenc}
\usepackage[french]{babel} 
\usepackage{fancyhdr}
\usepackage{amsthm}
\usepackage{amsmath}
\usepackage{parskip}
\usepackage{tikz}
\usetikzlibrary{patterns}
\usepackage{tkz-euclide}
\usepackage{mdframed}
\usepackage{xcolor}
\usepackage[margin=2cm]{geometry}

\newmdenv[linecolor=black,innertopmargin=2em,innerbottommargin=2em,backgroundcolor=lightgray!50!white]{infobox}

\setlength\headheight{24pt} 

\theoremstyle{definition}
\newtheorem*{dfns}{Définitions}
\newtheorem*{dfn}{Définition}
\newtheorem*{prop}{Propriétés}

\begin{document}
\pagestyle{fancy}
%... then configure it.
\fancyhead{} % clear all header fields
\fancyhead[L]{Angles \& Triangles}
\fancyhead[C]{\textbf{Leçon à compléter}}
\fancyhead[R]{\today}
\fancyfoot{} % clear all footer fields
\fancyfoot[C]{\thepage}
\section*{Nommer et reconnaître un angle}
Un angle se note avec trois points, par exemple, $\widehat{ABC}$. Le point du milieu correspond au \textbf{sommet} de l'angle, les deux autres points se trouvent sur les côtés de l'angle.

\ 
\begin{infobox}
\begin{dfns}\ 
 \begin{itemize}
  \item Un angle \textbf{aigu} est un angle qui mesure \dotfill
  \item Un angle \textbf{obtus} est un angle qui mesure \dotfill
  \item Un angle \textbf{droit} est un angle qui mesure \dotfill
  \item Un angle \textbf{plat} est un angle qui mesure \dotfill
  \item Deux angles sont \textbf{complémentaires} si la somme de leur mesure est \dotfill
  \item Deux angles sont \textbf{supplémentaires} si la somme de leur mesure est \dotfill
 \end{itemize}
\end{dfns}
\end{infobox}

Pour savoir si un angle est aigu, obtus, ou droit, on utilise une \textbf{équerre}. Pour connaître la mesure précise d'un angle, on utilise un \textbf{rapporteur}.

\section*{Nommer et reconnaître un triangle}
Un triangle se note avec trois points, par exemple, $ABC$. Les trois points peuvent être mis dans l'ordre que l'on veut.

Le triangle $ABC$ a trois angles : \dotfill

Le triangle $ABC$ a aussi trois côtés : \dotfill

\ 
\begin{infobox}
\begin{dfns}\ 
 \begin{itemize}
  \item Un triangle \textbf{rectangle} est un triangle qui a \dotfill
  \item Un triangle \textbf{isocèle} est un triangle qui a \dotfill
  \item Un triangle \textbf{équilatéral} est un triangle qui a \dotfill
  \item Un triangle \textbf{aigu} est un triangle qui a \dotfill
  \item Un triangle \textbf{obtus} est un triangle qui a \dotfill
 \end{itemize}
\end{dfns}
\end{infobox}
Pour savoir si un triangle est rectangle, on utilise une \textbf{équerre}, et pour savoir si un triangle est isocèle ou équilatéral, on utilise un \textbf{compas}.

\begin{minipage}{0.45\textwidth}
 \begin{center}
\begin{tikzpicture}
 \draw (0,0) node(A){}
  -- (4,0) node(C){}
  -- (2,4) node(B){}
  -- cycle;
  
  \draw (0,0) node ()[anchor=north]{$A$};
  \draw (4,0) node ()[anchor=north]{$C$};
  \draw (2,4) node ()[anchor=east]{$B$};
\end{tikzpicture}\\
Le triangle $ABC$ est isocèle en $B$. Les longueurs $AB$ et $CB$ sont égales. Les angles $\widehat{BAC}$ et $\widehat{BCA}$ sont égaux.
\end{center}
\end{minipage}\hfill\begin{minipage}{0.45\textwidth}
\begin{center}
\begin{tikzpicture}
 \draw (0,0) node(A){}
  -- (4,0) node(C){}
  -- (2,3.464) node(B){}
  -- cycle;
  \draw (0,0) node ()[anchor=north]{$D$};
  \draw (4,0) node ()[anchor=north]{$E$};
  \draw (2,3.464) node ()[anchor=south]{$F$};
\end{tikzpicture}\\
Le triangle $DEF$ est équilatéral. Ses trois côtés et ses trois angles sont égaux. Il a trois axes de symmétrie.
\end{center}
\end{minipage}

\section*{Construire un triangle}

Pour construire un triangle, il y a trois méthodes :


\begin{minipage}{0.55\textwidth}
\begin{infobox}
\subsection*{Avec la longueur des trois côtés}
\begin{enumerate}
 \item On trace un des trois côtés, celui que l'on veut.
 \item On reporte au compas les deux autres longueurs depuis les extrémités du côté déjà tracé.
 \item On trace les deux autres côtés en reliant les extrémités du côté déjà tracé au point d'intersection des arcs de cercle.
\end{enumerate}\end{infobox}
\end{minipage}\hfill\begin{minipage}{0.39\textwidth}
\begin{center}
               \begin{tikzpicture}
                \draw (0,0)--(0,4)--(8,4)--(8,0)--(0,0);
               \end{tikzpicture}
Tracer le triangle $GHI$ tel que $GH=2$cm, $HI=3$cm et $GI=4$cm.\end{center}
              \end{minipage}

Attention, cela ne marche que si les trois côtés vérifient \textbf{l'inégalité triangulaire} : dans un triangle, la somme des longueurs de deux côtés est toujours plus grande que le troisième côté.
\begin{minipage}{0.55\textwidth}\begin{infobox}
\subsection*{Avec deux longueurs et un angle}
\begin{enumerate}
 \item On trace un des deux côtés dont on connaît la longueur.
 \item On trace l'angle que l'on connaît.
 \item On trace le sommet restant en mesurant la longueur et on le relie à l'autre extrémité.
\end{enumerate}\end{infobox}
\end{minipage}\hfill\begin{minipage}{0.39\textwidth}
\begin{center}
               \begin{tikzpicture}
                \draw (0,0)--(0,4)--(8,4)--(8,0)--(0,0);
               \end{tikzpicture}
Tracer le triangle $JKL$ tel que $JK=2$cm, $KL=3$cm, et $\widehat{JKL}=25^\circ$.\end{center}
              \end{minipage}
              
              \ 
              

\begin{minipage}{0.55\textwidth}\begin{infobox}
\subsection*{Avec une longueur et deux angles}
\begin{enumerate}
 \item On trace le côté dont on connaît la longueur.
 \item On trace les deux angles que l'on connaît.
 \item Le troisième sommet est donné par le point d'intersection.
\end{enumerate}\end{infobox}
\end{minipage}\hfill\begin{minipage}{0.39\textwidth}
\begin{center}
               \begin{tikzpicture}
                \draw (0,0)--(0,4)--(8,4)--(8,0)--(0,0);
               \end{tikzpicture}
Tracer le triangle $MNO$ tel que $MN=3$cm, $\widehat{MNO}=45^\circ$ et $\widehat{NMO}=30^\circ$.\end{center}
              \end{minipage}

\section*{Le cercle circonscrit}
\begin{minipage}{0.55\textwidth}
\begin{infobox}
\begin{dfn}\ 
La \textbf{médiatrice} d'un segment est la droite perpendiculaire au segment et qui passe par son milieu.
\end{dfn}

\ 

Pour la tracer, on trace à chacune des extrémités du segment un arc de cercle de même rayon, et on trace la droite qui passe par les deux intersections.
\end{infobox}
\end{minipage}\hfill\begin{minipage}{0.39\textwidth}
\begin{center}
               \begin{tikzpicture}
                \draw (0,0)--(0,4)--(8,4)--(8,0)--(0,0);
                \draw (2,1) node(A)[anchor=north]{$P$} -- (6,1) node(B)[anchor=north]{$Q$};
                \draw (2,0.9)--(2,1.1);
                \draw (6,0.9)--(6,1.1);
               \end{tikzpicture}
               Tracer la médiatrice du segment $[PQ]$
               \end{center}
              \end{minipage}
              
              \ 
              
  \begin{minipage}{0.55\textwidth}            
\begin{infobox}
\begin{dfn}\ 
 Le \textbf{cercle circonscrit} à un triangle est le cercle qui passe par ses trois sommets.
 \end{dfn}
 
 \ 

Pour le tracer, on trace la médiatrice de chacun des côtés. Les trois droites se croisent en un point : c'est le centre du cercle circonscrit. On place son compas sur ce point et on trace le cercle qui passe par l'un des sommets.

\end{infobox}\end{minipage}\hfill\begin{minipage}{0.39\textwidth}
\begin{center}
               \begin{tikzpicture}
                \draw (0,0)--(0,4)--(8,4)--(8,0)--(0,0);
                \draw (1.5,0.7) node(A)[anchor=north]{$S$} -- (7,1) node(B)[anchor=north]{$T$} -- (4.2,3.1) node(C)[anchor=south]{$U$} -- cycle;
               \end{tikzpicture}
               Tracer le cercle circonscrit du triangle $SUT$.
               \end{center}
              \end{minipage}

              Le centre du cercle circonscrit se trouve :
              \begin{itemize}
 \item À l'intérieur du triangle si le triangle est aigu,
 \item À l'extérieur du triangle si le triangle est obtus,
 \item Au milieu de \textbf{l'hypothénuse} du triangle si le triangle est droit.
\end{itemize}

\ 

L'hypothénuse est le nom que l'on donne au côté opposé à l'angle droit dans un triangle rectangle.
              
\end{document}
