\documentclass[a4paper,12pt]{exam}
%Documentation de la classe exam : https://mirrors.ircam.fr/pub/CTAN/macros/latex/contrib/exam/examdoc.pdf
\usepackage[utf8]{inputenc}
\usepackage[T1]{fontenc}
\usepackage[french]{babel} 
\usepackage{amsmath,amssymb}
\usepackage{titlesec}
\usepackage{tikz}
\usepackage{lastpage}

\DeclareMathOperator{\cm}{cm}


%En-tête et pied de page
\pagestyle{headandfoot}
\header{Classe :\\Nom \& Prénom :}{\Large\textbf{Évaluation}}{Page \thepage/\pageref{LastPage}\\\today}
\headrule
\footrule
\footer{}{}{}

\titleformat{\section}[frame]{\Huge\bfseries\filright}{}{2mm}{\centering  }

\printanswers					%Commenter cette ligne pour cacher les solutions.

\qformat{\textbf{Exercice \thequestion~:}\hfill}
\newcommand{\changesolution}[1]{\renewcommand{\solutiontitle}{\noindent\textbf{#1~:}}
}
\changesolution{Figures}

\extrawidth{1cm}

\begin{document}
\section{Angles \& Triangles}

\begin{questions}
% Coller ici le code des questions issues de DAM.
\question\ 
\begin{enumerate}
\item Construire un triangle $ABC$ tel que $AB=4\cm$, $BC=3 \cm$, et $AC=5\cm$.
\item Construire un triangle $DEF$ tel que $DE=3\cm$, $EF=3\cm$ et $\widehat{DEF}=110^\circ$.
\item Construire un triangle $GHI$ tel que $GH=4\cm$, $\widehat{GHI}=60^\circ$ et $\widehat{HGI}=60^\circ$.
\end{enumerate}

\begin{solution}
\ \\\ \\\ \\\ \\\ \\\ \\\ \\\ \\\ \\\ \\\ \\\ \\\ \\\ \\\ \\
\end{solution}

\question
Décrire chaque triangle de l'exercice 1 : dire s'il est rectangle, aigu, isocèle... 

\changesolution{Réponse}
\begin{solution}
\ \\\ \\\ \\\ \\\ \\\ \\\ \\\\ \\\ \\\ \\\ \\\ \\\ \\\ \\
\end{solution}

\question
Tracer le cercle circonscrit du triangle ci-dessous : 
\changesolution{Figure}\begin{solution}\ \\\ \\\ \\\\ \\\ \\\ \\\

\begin{center}
\begin{tikzpicture}
                 \draw (0,0)  -- (-1,8)  -- (4,6)  -- cycle;
                \end{tikzpicture}
                \end{center}
                \ \\\ \\\ \\\
                \ \\\ \\\ \\\
\end{solution}

\question\
Peut-on construire un triangle dont les côtés mesurent $2\cm$, $3\cm$ et $7\cm$ ? Pourquoi ?
\changesolution{Réponse}
\begin{solution}
\ \\\ \\\ \\\ \\\ \\\ \\\ \\
\end{solution}
\end{questions}\begin{questions}

\section{Multiples \& Diviseurs}
\question\ 
Entourez la ou les bonnes réponses.
\setlength{\tabcolsep}{15pt}
\renewcommand{\arraystretch}{1.5}
\begin{center}
\begin{tabular}{|l||c|c|c|}\hline
 Le nombre 6 est un diviseur de & 2 & 12 & 15\\\hline
 Le nombre 28 est un multiple de & 2 & 3 & 4\\\hline
 Le nombre 5932 est divisible par & 2 & 3 & 4\\\hline
 2818 est égal à & $2 \times 1409$ & $2\times 1409+1$ & $4\times717$\\\hline 
\end{tabular}
\end{center}

\ 

\question
Il y a 60 minutes dans une heure : ce nombre n'a pas été choisi au hasard, c'est un nombre très pratique pour mesurer le temps car il a beaucoup de diviseurs. On peut le diviser par 2 pour faire des demi-heures, par 4 pour faire des quarts d'heure...

\

Pouvez-vous lister tous les diviseurs de 60 ? N'oubliez pas que les diviseurs vont par paire !

\changesolution{Réponse}
\begin{solution}
\ \\\ \\\ \\\ \\\ \\\ \\\ \\\ \\\ \\\ \\\ \\\ \\\ \\\ \\\ \\\ \\\ \\\ \\\ \\\ \\\ \\\ \\\ \\\ \\\ \\ 
\end{solution}

\end{questions}
\end{document}
