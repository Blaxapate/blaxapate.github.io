\documentclass[a4paper,12pt]{exam}
%Documentation de la classe exam : https://mirrors.ircam.fr/pub/CTAN/macros/latex/contrib/exam/examdoc.pdf
\usepackage[utf8]{inputenc}
\usepackage[T1]{fontenc}
\usepackage[french]{babel} 
\usepackage{amsmath,amssymb}
\usepackage{titlesec}
\usepackage{lastpage}

\DeclareMathOperator{\cm}{cm}

%En-tête et pied de page
\pagestyle{headandfoot}
\header{Nom :\\Classe :}{\Large\textbf{Angles \& Triangles}}{\today\\page \thepage/\pageref{LastPage}}
\headrule
\footrule
\footer{}{Page \thepage}{}

\titleformat{\section}[frame]{\Huge\bfseries\filright}{}{2mm}{\centering  }

\printanswers					%Commenter cette ligne pour cacher les solutions.

\qformat{\textbf{Exercice \thequestion~:}\hfill}
\newcommand{\changesolution}[1]{\renewcommand{\solutiontitle}{\noindent\textbf{#1~:}}
}
\changesolution{Figures}

\extrawidth{1cm}

\begin{document}
\section{Feuille d'exercices 1}

\begin{questions}
% Coller ici le code des questions issues de DAM.
\question\ 
\begin{enumerate}
\item\begin{enumerate}
\item Construire un triangle $ABC$ tel que $AB=5\cm$, $BC=3
\cm$, et $AC=3\cm$.
\item Construire un triangle $DEF$ tel que $DE=5\cm$, $EF=3\cm$ et $DF=4\cm$.
\item Construire un triangle $GHI$ tel que $GH=4\cm$, $HI=4\cm$
et $GI=4\cm$.
\end{enumerate}

\begin{solution}
\ \\\ \\\ \\\ \\\ \\\ \\\ \\\
\end{solution}
\item Peut-on construire un triangle dont les côtés mesurent $1\cm, 2\cm$ et $4\cm$ ? Pourquoi ?
\changesolution{Réponse}
\begin{solution}
\ \\\ \\
\end{solution}
\end{enumerate}

\question\ 
\begin{enumerate}
\item\begin{enumerate}
\item Construire un triangle $JKL$ tel que $JK=5\cm$, $KL=3
\cm$, et $\widehat{JKL}=30^\circ$.
\item Construire un triangle $MNO$ tel que $\widehat{MNO}=25^\circ$, $\widehat{MON}=40^\circ$, et $ON=4\cm$.
\item Construire un triangle $PQR$ tel que $\widehat{PQR}=110^\circ$, $QR=2\cm$, et $PR=4\cm$.
\end{enumerate}
\changesolution{Figures}
\begin{solution}
\ \\\ \\\ \\\ \\\ \\\ \\\ \\\
\end{solution}
\item Peut-on construire un triangle avec deux angles droits ? Pourquoi ? 
\changesolution{Réponse}
\begin{solution}
\ \\\ \\
\end{solution}
\end{enumerate}
\question
Parmi les triangles des exercices 1 et 2, lesquels sont isocèles ? Lesquels sont droits ? Lesquels sont aigus ?
\changesolution{Réponse}
\begin{solution}
\ \\\ \\\ \\\ \\\ \\\ \\\ \\\
\end{solution}

\question\ 
\begin{enumerate}
 \item Tracer un segment $AB$ de longueur $5\cm$, puis, à l'aide du compas, tracer deux cercles de rayon $3\cm$ et de centres $A$ et $B$. Tracer la droite qui relie les deux points d'intersection de ces deux cercles. Cette droite s'appelle la \textbf{médiatrice} du segment $AB$. Que remarque-t-on ?
 \item Placer un troisième point $C$ puis tracer la médiatrice des segments $AC$, et $BC$. Que remarque-t-on ?
\item On appelle $O$ le point d'intersection des médiatrices. Tracer un cercle de centre $O$ passant par $A$. Ce cercle s'appelle le cercle \textbf{circonscrit} du triangle $ABC$. Que remarque-t-on ?
\end{enumerate}
\changesolution{Figures}
\begin{solution}
\ \\\ \\\ \\\ \\\ \\\ \\\ \\\ \\\ \\\ \\\ \\\ \\\ \\
\end{solution}
\changesolution{Réponses}
\begin{solution}
\begin{enumerate}
\item\ 
\item\ 
\item\ 
\end{enumerate}
\end{solution}


\end{questions}
\end{document}
