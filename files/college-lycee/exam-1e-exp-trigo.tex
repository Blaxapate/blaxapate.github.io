\documentclass[a4paper,12pt]{exam}
%Documentation de la classe exam : https://mirrors.ircam.fr/pub/CTAN/macros/latex/contrib/exam/examdoc.pdf
\usepackage[utf8]{inputenc}
\usepackage[T1]{fontenc}
\usepackage[french]{babel} 
\usepackage{amsmath,amssymb}
\usepackage{titlesec}
\usepackage{tikz}
\usepackage{lastpage}
\usepackage{pgfplots}
\usepackage{enumerate}
\usepackage[inline]{enumitem}
\usepackage{tkz-tab}

\DeclareMathOperator{\cm}{cm}

%En-tête et pied de page
\pagestyle{headandfoot}
\header{1\textsuperscript{ère} maths spé}{\Large\textbf{Évaluation} -- Exponentielle, trigonométrie}{%
page \thepage/2
%LLG 2023-24
}
\headrule
\footer{}{}{}

\titleformat{\section}[frame]{\Huge\bfseries\filright}{}{2mm}{\centering  }

%\printanswers					%Commenter cette ligne pour cacher les solutions.


\newcommand{\changesolution}[1]{\renewcommand{\solutiontitle}{\noindent\textbf{#1~:}}
}
\changesolution{Corrigé}

%command \exercice to be called later
\newcommand{\exercice}[2]{\qformat{\textbf{Exercice \thequestion}#1\hfill\textbf{#2}}\question\ }
%command \exo for a blank exercise
\newcommand{\exo}{\exercice{}{\ }}
%command \exop{x} for an exercise worth x points
\newcommand{\exop}[1]{\exercice{}{#1pt}}
%command \exon{x} for an exercise named x
\newcommand{\exon}[1]{\exercice{\emph{, #1.}}{\ }}
%commmand \exonp{x,y} for an exercise named x and worth y points
\newcommand{\exonp}[2]{\exercice{\emph{, #1.}}{#2pt}}

\extrawidth{1cm}

\begin{document}
\begin{questions}
\exonp{tiré d'un sujet de bac 2015}{6}
La fonction $f$ est définie sur $\mathbb R$ par $f(x)=x^2e^{-2x}$.
\begin{enumerate}\item Justifier que pour tout réel $x$, $f'(x)=2xe^{-2x}(1-x)$.
\item Étudier les variations de la fonction $f$ sur $\mathbb R$.
\item Déterminer la limite de $f$ en $-\infty$.
\item Vérifier que pour tout réel $x$, $f(x)=(\tfrac x{e^x})^2$. Conjecturer la limite de $f(x)$ quand $x$ tend vers $+\infty$.
\item On définit la fonction $g$ sur $\mathbb R$ par $g(x)=x^2e^{-x}$. Calculer $g'(x)$ et montrer que sur $]2,+\infty]$, $g$ est décroissante et positive. On en déduit que $g$ converge vers une valeur réelle positive ou nulle lorsque $x$ tend vers l'infini.
\item En déduire que $\tfrac{g(x)}x$ tend vers $0$ lorsque $x$ tend vers $+\infty$, et déterminer la limite de $f(x)$ en $+\infty$.
\end{enumerate}
 \begin{solution}

 \end{solution}

\ \vfill
\exonp{La loi des sinus}{3}
On considère un triangle $ABC$, on note $a=BC$, $b=AC$ et $c=AB$, ainsi que $\alpha=\widehat{BAC}$, $\beta=\widehat{ABC}$ et $\gamma=\widehat{ACB}$.
\begin{enumerate}
 \item Soit $H$ le projeté orthogonal de $A$ sur $BC$, on note $h=AH$. Exprimer $h$ en fonction de $\sin(\beta)$ et en fonction de $\sin(\gamma)$.
 \item En déduire $\tfrac {\sin(\beta)}b=\tfrac{\sin(\gamma)}c$.
 \item En déduire la loi des sinus : $\tfrac{\sin(\alpha)}a=\tfrac{\sin(\beta)}b=\tfrac{\sin(\gamma)}c.$
 
\end{enumerate}

\ \vfill
\exop{3}
Parmi les fonctions suivantes, lesquelles sont paires ? Lesquelles sont impaires ? Lesquelles sont périodiques ?
\begin{enumerate}
\begin{minipage}{0.33\textwidth}
 \item $f(x)=e^{(x^2)}\cos(x)$
\end{minipage}
\begin{minipage}{0.33\textwidth}
 \item $g(x)=e^x-e^{-x}$
\end{minipage}
\begin{minipage}{0.33\textwidth}
 \item $h(x)=\sin(x)+e^{\sin(2x)}$
 \end{minipage}
\end{enumerate}
\ \vfill
\exonp{Ptolémée et le pentagone}{8}
\begin{enumerate}
\item On se place dans un repère orthonormé d'origine $O$. Soient $\alpha$ et $\beta$ deux réels tels que $0<\alpha<\beta<\pi$ et soient $A$ et $B$ les points correspondants sur le cercle trigonométrique. Soient $C$ et $D$ les points de coordonnées respectives $(-1\;;0)$ et $(1\;;0)$.
\begin{enumerate}
 \item On considère les triangle $OAB$. Soit $H$ le milieu de $[AB]$. Montrer que $\widehat{HOB}=\tfrac{\beta-\alpha}2$ et en déduire que la longueur $AB$ est égale à $2\sin(\tfrac{\beta-\alpha}2)$.
 \item Exprimer de même les longueurs $BC$, $CD$, $AD$, $AC$ et $BD$.
 \item Montrer que $\tfrac{AB\times CD}4=\sin(\tfrac{\beta-\alpha}2)$, $\tfrac{AD\times BC}4=\sin(\tfrac\alpha2)\cos(\tfrac\beta2)$ et $\tfrac{AC\times BD}4=\cos(\tfrac\alpha2)\sin(\tfrac\beta2)$.
 \item En déduire la formule de Ptolémée dans ce cas : $$AC\times BD=AB\times CD+BC\times AD.$$
\end{enumerate}
 \item On admet la formule de Ptolémée dans un cas plus général : si $A$, $B$, $C$ et $D$ sont quatre points sur cercle trigonométrique dans cet ordre, alors $$AC\times BD=AB\times CD+BC\times AD.$$ Soit $A$, $B$, $C$, $D$ et $E$ les points correspondants du cercle trigonométriques correspondants respectivement aux réels $0$, $\tfrac{2\pi}5$, $\tfrac{4\pi}5$, $\tfrac{6\pi}5$ et $\tfrac{8\pi}5$. 
 \begin{enumerate}
  \item Tracer la figure et justifier que $ABCDE$ est un pentagone régulier, on notera $a=AB$ la longueur de son côté.
  \item On note $d=AC$ la longueur d'une diagonale de $ABCDE$. En utilisant la formule de Ptolémée, montrer que $d^2-ad-a^2=0$.
  \item On note $\varphi$ l'unique solution positive de l'équation $x^2-x-1=0$. Montrer que $x=a\varphi$ est solution de l'équation $x^2-ax-a^2=0$. Donner la valeur exacte de $\varphi$.
  \item Soit $H$ le milieu de $[AC]$. En considérant le triangle $AHB$, et en justifiant toutes vos étapes, montrer que $\cos(\tfrac\pi5)=\tfrac{1+\sqrt5}4$, et en déduire la valeur exacte de $\sin(\tfrac\pi5)$.
 \end{enumerate}
\end{enumerate}

\end{questions}
\end{document}
