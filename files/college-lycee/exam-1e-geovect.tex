\documentclass[a4paper,12pt]{exam}
%Documentation de la classe exam : https://mirrors.ircam.fr/pub/CTAN/macros/latex/contrib/exam/examdoc.pdf
\usepackage[utf8]{inputenc}
\usepackage[T1]{fontenc}
\usepackage[french]{babel} 
\usepackage{amsmath,amssymb}
\usepackage{titlesec}
\usepackage{pgf,tikz}
\usetikzlibrary{arrows}
\usepackage{lastpage}
\usepackage{pgfplots}
\usepackage{enumerate}
\usepackage[inline]{enumitem}

\DeclareMathOperator{\cm}{cm}
\DeclareMathOperator{\m}{m}
\newcommand{\vect}{\overrightarrow}


%En-tête et pied de page
\pagestyle{headandfoot}
\header{1\textsuperscript{ère} maths spé}{\Large\textbf{Évaluation} -- Géométrie vectorielle}{
%page \thepage/2
\today
%LLG 2023-24
}
\headrule
\footer{}{}{}

\titleformat{\section}[frame]{\Huge\bfseries\filright}{}{2mm}{\centering  }

%\printanswers					%Commenter cette ligne pour cacher les solutions.


\newcommand{\changesolution}[1]{\renewcommand{\solutiontitle}{\noindent\textbf{#1~:}}
}
\changesolution{Corrigé}

%command \exercice to be called later
\newcommand{\exercice}[2]{\qformat{\textbf{Exercice \thequestion}#1\hfill\textbf{#2}}\question\ }
%command \exo for a blank exercise
\newcommand{\exo}{\exercice{}{\ }}
%command \exop{x} for an exercise worth x points
\newcommand{\exop}[1]{\exercice{}{#1pt}}
%command \exon{x} for an exercise named x
\newcommand{\exon}[1]{\exercice{\textbf{ :}\emph{ #1.}}{\ }}
%commmand \exonp{x,y} for an exercise named x and worth y points
\newcommand{\exonp}[2]{\exercice{\textbf{ :}\emph{ #1.}}{#2pt}}

\extrawidth{1cm}

\begin{document}
\begin{questions}
\exop{10}
\begin{minipage}{0.5\textwidth}
 
Sur la figure ci-contre, $(O,\vect{OU},\vect{OV})$ forment un repère orthonormé, $U$ est le milieu de $[OA]$, $V$ le milieu de $[OB]$, et $BOAT$ est un carré.
\begin{enumerate}
 \item\begin{enumerate}\item Justifier que $(BU)$ est la droite d'équation $2x+y=2$. \item Donner une équation cartésienne de la droite $(TV)$. \item Déterminer les coordonnées de $S$, leur point d'intersection.
 \end{enumerate}
\end{enumerate} 
\end{minipage}
\begin{minipage}{0.5\textwidth}
\begin{center}
 \begin{tikzpicture}
  \draw (0,0)--(4,0)--(4,4)--(0,4)--cycle;
  \node at (0,0) [anchor=north east] {$O$};
  \node at (4,0) [anchor=north west] {$A$};
  \node at (0,4) [anchor=south east] {$B$};
  \node at (4,4) [anchor=south west] {$T$};
  \node at (0,2) [anchor=east] {$V$};
  \node at (2,0) [anchor=north] {$U$};
  \draw (2,-0.1)--(2,0.1);
  \draw (-0.1,2)--(0.1,2);
  
 \end{tikzpicture}
\end{center}
\end{minipage}
\begin{enumerate}\stepcounter{enumi}
 \item Soit $(d)$ la droite perpendiculaire à $(TV)$ passant par $A$.
 \begin{enumerate}
 \item Donner un vecteur normal à $(d)$.
 \item En déduire une équation cartésienne de $(d)$. \item Donner les coordonnées de $P$, le point d'intersection de $(d)$ et $(TV)$.
 \end{enumerate}
 \item Soit $(d')$ la droite parallèle à $(TV)$ passant par $O$.
 \begin{enumerate}\item Donner un vecteur normal à $(d')$.
 \item En déduire une équation cartésienne de $(d')$.
 \item Vérifier que les points $H\colon(\tfrac45\;;\tfrac25)$ et $I\colon(\tfrac85\;;\tfrac45)$ sont sur la droite $(d')$.
\end{enumerate}
\item\begin{enumerate}
\item Montrer par le calcul que $\vect{BU}\cdot\vect{TV}=0$ et $\vect{SI}\cdot\vect{HP}=0$.
\item En déduire que $SHIP$ est un carré, puis montrer que l'aire de $BOAT$ est cinq fois l'aire de $SHIP$.
\end{enumerate}
\end{enumerate}

\exonp{Inversion circulaire}5
On se place dans un repère orthonormé $(O,\vec\imath,\vec\jmath)$.
\begin{enumerate}
 \item Donner une équation cartésienne du cercle $\mathcal C$ de centre $O$ et de rayon 1.
 \item Soit $P\colon(a\;;b)$ un point tel que $a^2+b^2>1$. Est-il à l'intérieur ou à l'extérieur de $\mathcal C$ ?
 \item Donner une équation vectorielle du cercle $\mathcal C'$ de diamètre $OP$.
 \item Soient $M$ et $N$ les points d'intersections de $\mathcal C$ et $\mathcal C'$. Montrer que $\vect{OM}\cdot\vect{OP}=1$ et $\vect{ON}\cdot\vect{OP}=1$.
 \item Soit $P'$ le milieu de $[MN]$. Exprimer $\vect{OP'}$ en fonction de $\vect{OM}$ et $\vect{ON}$, puis montrer que $\vect{OP}\cdot\vect{OP'}=1$.
\end{enumerate}

\exop5
Dans un repère orthonormé $(O,\vec\imath,\vec\jmath)$, on considère le point $F\colon(0\;;1)$.

Décrire l'ensemble des points $M\colon(x\;;y)$ qui satisfont l'équation $\vect{OM}\cdot\vect{OF}=||\vect{FM}||$.

\exon{Bonus}
Nous sommes 40 dans cette salle de classe.
\begin{enumerate}
 \item Quelle est la probabilité que l'un·e d'entre nous fête son anniversaire aujourd'hui ?
 \item Quelle est la probabilité que deux d'entre nous fêtent leur anniversaire aujourd'hui ?
 \item Sachant qu'au moins une personne dans cette classe fête son anniversaire aujourd'hui, quelle est la probabilité qu'au moins deux personnes fêtent leur anniversaire aujourd'hui ?
 \item Combien de personnes faut-il au minimum dans une salle pour que la probabilité que deux personnes fêtent leur anniversaire le même jour soit >50\% ?
\end{enumerate}

\end{questions}
\end{document}
