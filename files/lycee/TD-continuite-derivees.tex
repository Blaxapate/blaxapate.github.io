\documentclass[a4paper,12pt]{exam}
%Documentation de la classe exam : https://mirrors.ircam.fr/pub/CTAN/macros/latex/contrib/exam/examdoc.pdf
\usepackage[utf8]{inputenc}
\usepackage[T1]{fontenc}
\usepackage[french]{babel} 
\usepackage{amsmath,amssymb}
\usepackage{titlesec}
\usepackage{tikz}
\usepackage{lastpage}
\usepackage{pgfplots}
\usepackage{enumerate}
\usepackage[inline]{enumitem}

\DeclareMathOperator{\cm}{cm}

%En-tête et pied de page
\pagestyle{headandfoot}
\header{}{\Large\textbf{Travaux Dirigés -- correction}\\Continuité et dérivabilité}{
%\today
LLG 2023-24
}
\headrule
\footer{}{}{}

\titleformat{\section}[frame]{\Huge\bfseries\filright}{}{2mm}{\centering  }

\printanswers					%Commenter cette ligne pour cacher les solutions.

\qformat{\textbf{Exercice \thequestion~:}\hfill}
\newcommand{\changesolution}[1]{\renewcommand{\solutiontitle}{\noindent\textbf{#1~:}}
}
\changesolution{Corrigé}

\extrawidth{1cm}

\begin{document}
\begin{questions}

\question\ 
 Pour $n\in\mathbb N$, on considère la fonction $f_n$ définie sur $\mathbb R$ par :
 
 $$f_n(x)=\left\{\begin{array}{lc}x^n \sin(\tfrac{1}{x})&\text{si }x\neq 0\\0&\text{si }x=0\end{array}\right.$$
 
 \begin{enumerate}
  \item Montrez que $f_0$ est continue sur $\mathbb R^*$ mais discontinue en $0$.
  \item Montrez que $f_1$ est continue sur $\mathbb R$, dérivable sur $\mathbb R^*$, mais n'est pas dérivable en $0$.
  \item Montrez que $f_2$ est continue sur $\mathbb R$, dérivable sur $\mathbb R$, mais que sa dérivée n'est pas continue en $0$.
  \item Montrez que $f_3$ est continue sur $\mathbb R$, dérivable sur $\mathbb R$, et que sa dérivée est continue sur $\mathbb R$.
 \end{enumerate}
 
 \begin{solution}\ 
 Fixons $n$. On considère les fonctions $u: x\rightarrow x^n$, $v: x\rightarrow \tfrac1x$ et $w: x\rightarrow\sin(x)$. $v$ est continue sur $\mathbb R^*$ et $w$ est continue sur $\mathbb R$, donc leur composée $w\circ v$ est continue sur $\mathbb R^*$. $u$ est continue sur $\mathbb R$, donc le produit $u\times(w\circ v)$ est continu sur $\mathbb R^*$.
 
 Ainsi, $f_n$ est continue sur $\mathbb R^*$ quelque soit $n$. Le même argument montre que $f_n$ est dérivable sur $\mathbb R^*$.
 
 \begin{enumerate}
  \item Le fonction $f_0$ est continue sur $\mathbb R^*$ par l'argument ci-dessus. On considère désormais la suite $(u_n)_{n\in\mathbb N}$ définie par $u_n=(\tfrac{\pi}{2}+2n\pi)^{-1}$. On a $f_0(u_n)=\sin(\tfrac{\pi}{2}+2n\pi)=1$. Ainsi la suite $(u_n)_{n\in\mathbb N}$ tend vers $0$ et la suite $(f_0(u_n))_{n\in\mathbb N}$ tend vers $1$; or $f_0(0)=0\neq1$, donc $f_0$ n'est pas continue en $0$.
  \item La fonction $f_1$ est continue et dérivable sur $\mathbb R^*$ d'après l'argument ci-dessus. Soit $(u_n)_{n\in\mathbb N}$ une suite qui converge vers $0$, on souhaite montrer que $(f_1(u_n))_{n\in\mathbb N}$ converge vers $f_1(0)=0$. Si $u_n=0$, alors $f_1(u_n)=0$, et en particulier $-|u_n|\leqslant f_1(u_n)\leqslant|u_n|$. Si $u_n\neq0$, $f_1(u_n)=u_n\sin(\tfrac{1}{u_n})$, et comme $\sin(x)\in[-1,1]$ pour tout $x$, on a $-|u_n|\leqslant f_1(u_n)\leqslant|u_n|$. D'après le théorème des gendarmes, $(f_1(u_n))_{n\in\mathbb N}$ tend vers $0$, et donc $f_1$ est continue en $0$.  
  
  On considère désormais le taux de variation de $f_1$ en $0$:
  $$\frac{f_1(0+h)-f_1(0)}{h}=\frac{h\sin(\tfrac{1}{h})}{h}=\sin(\tfrac1h).$$
  Ce taux de variation n'a pas de limite quand $h$ tend vers 0, donc la fonction $f_1$ n'est pas dérivable en $0$. 
  \item On a pour tout $x\in\mathbb R$: $f_2(x)=xf_1(x)$. Comme $f_1$ est continue sur $\mathbb R$, $f_2$ est continue sur $\mathbb R$. De même $f_2$ est dérivable sur $\mathbb R^*$. On considère le taux de variation de $f_2$ en $0$:
  $$\frac{f_2(0+h)-f_2(0)}{h}=\frac{h^2\sin(\tfrac{1}{h})}{h}=h\sin(\tfrac1h).$$
  Ce taux de variation tend vers $0$ quand $h$ tend vers 0, donc $f_2$ est dérivable en $0$ et on a :
  $$f_2'(x)=\left\{\begin{array}{lc}2x \sin(\tfrac{1}{x})-\cos(\tfrac1x)&\text{si }x\neq 0\\0&\text{si }x=0\end{array}\right.$$
  Un argument similaire à la question 1 montre que $f_2'$ n'est pas continue en 0.
  \item On a pour tout $x\in\mathbb R$: $f_3(x)=xf_2(x)$. Comme $f_2$ est continue et dérivable sur $\mathbb R$, $f_3$ est continue et dérivable sur $\mathbb R$, et on a
  $$f_3'(x)=\left\{\begin{array}{lc}3x^2 \sin(\tfrac{1}{x})-x\cos(\tfrac1x)&\text{si }x\neq 0\\0&\text{si }x=0\end{array}\right.$$
  Un argument similaire à la question 2 montre que la fonction $x\rightarrow x\cos(\tfrac1x)$ est continue en $0$ et donc que $f_3'$ est continue en 0. 
 \end{enumerate}
 \end{solution}
 \question Démontrez qu'il existe deux points antipodaux (diamétralement opposées) de l'équateur où il fait la même température. Explicitez vos hypothèses.
 
 \begin{solution}\ 
  On considère la fonction $T$ définie sur $\mathbb R$ qui associe à $x$ la température à l'équateur la longitude $x$ (mesurée en degrée, modulo 360). On suppose que la fonction $T$ est continue car elle représente une grandeur physique.
  
  On considère la fonction $f$ définie par $f(x)=T(x)-T(x+180)$, la différence de température entre le point de longitude $x$ et son point antipodal. On veut montrer que $f$ s'annule en un point. Si $f(0)=0$, c'est fait. Sinon, comme $f(0)=-f(180)$, si l'un des deux est strictement positif, l'autre est srtictement négatif. Comme $f$ est continue, il existe $x\in[0,180]$ tel que $f(x)=0$.
 \end{solution}

 
\end{questions}
\end{document}
