\usepackage{amssymb}
\usepackage{dsfont}
\newcommand{\negphantom}[1]{\settowidth{\dimen0}{#1}\hspace*{-\dimen0}}
\usepackage{enumerate}
\usepackage{mathtools}
\usepackage{amsmath}
\usepackage{amsthm}
\usepackage{color}
\usepackage{bbm}         % Q,Z,R etc.
\usepackage{amsfonts}
\usepackage{graphicx,hyperref,verbatim}
\usepackage{url}
\usepackage{tikz,multicol,manfnt}
\usepackage[weather,clock,alpine]{ifsym}
\usepackage{xparse}
\usepackage{parnotes}

%This is because of my weird habits
\newcommand{\rond}{\mathcal}
\newcommand{\urg}[1]{\textcolor{red}{#1}}
\newcommand{\goth}{\mathfrak}
\newcommand{\lagnle}{\langle}
\newcommand{\leqlsant}{\leqslant}
\newcommand{\espilon}{\epsilon}
\newcommand{\indu}{\vdash}
\newcommand{\satisf}{\vDash}
\newcommand{\et}{\wedge}
\newcommand{\ou}{\vee}
\newcommand{\Et}{\bigwedge}
\newcommand{\Ou}{\bigvee}

%Those are universals except when they are not
\newcommand{\inject}{\hookrightarrow}
\newcommand{\surject}{\twoheadrightarrow}
\newcommand{\comm}[1]{\left[#1,#1\right]}
\newcommand{\floor}[1]{\left \lfloor{#1}\right \rfloor} 
\newcommand{\norm}[1]{\left\Vert #1\right\Vert}
\newcommand{\abs}[1]{\left\vert #1\right\vert}
\newcommand{\leqcur}{\preccurlyeq}
\newcommand{\geqcur}{\succcurlyeq}
\NewDocumentCommand{\set}{mg}{\left\{#1\IfNoValueF{#2}{\;\middle\vert\;#2}\right\}}

%To change the footnote
\renewcommand{\thefootnote}{\fnsymbol{footnote}}

%A large ammount of operators, some useful
\DeclareMathOperator{\Aff}{Aff}     % affine Gruppe
\DeclareMathOperator{\Aut}{Aut}     % Automorphismen-Gruppe
\DeclareMathOperator{\bin}{bin}
\DeclareMathOperator{\Br}{Br}
\DeclareMathOperator{\Char}{char}
\DeclareMathOperator{\coker}{coker} % Cokern
\DeclareMathOperator{\diag}{diag}   % Diagonalmatrix
\DeclareMathOperator{\End}{End}     % Endomorphismen eines Vektorraumes
\DeclareMathOperator{\Ext}{Ext}
\DeclareMathOperator{\Fun}{Fun}
\DeclareMathOperator{\Gal}{Gal}
\DeclareMathOperator{\GL}{GL}
\DeclareMathOperator{\ggT}{ggT}     % ggT
\DeclareMathOperator{\Hom}{Hom}
\DeclareMathOperator{\id}{id}       % Identität
\DeclareMathOperator{\im}{im}       % Bild (image)
\DeclareMathOperator{\Ker}{Ker}
\DeclareMathOperator{\kgV}{kgV}     % kgV
\DeclareMathOperator{\Ob}{Ob}
\DeclareMathOperator{\On}{On}
\DeclareMathOperator{\ord}{ord}
\DeclareMathOperator{\Out}{Out}
\DeclareMathOperator{\PGL}{PGL}     % projektive lineare Gruppe
\DeclareMathOperator{\PSL}{PSL}
\DeclareMathOperator{\Res}{Res}     % Residuenabbildung
\DeclareMathOperator{\rk}{rk}       % Rang einer Matrix
\DeclareMathOperator{\sgn}{sgn}     % Signum einer Permutation
\DeclareMathOperator{\SL}{SL}
\DeclareMathOperator{\SO}{SO}       % spezielle orthogonale Gruppe
\DeclareMathOperator{\supp}{supp}   % Träger eines Polynoms
\DeclareMathOperator{\Sym}{Sym}     % symmetrische Gruppe
\DeclareMathOperator{\Th}{Th}
\DeclareMathOperator{\Tor}{Tor}
\DeclareMathOperator{\tr}{tr}       % Spurabbildung
\DeclareMathOperator{\trdeg}{trdeg}
\DeclareMathOperator{\Diag}{Diag}
\DeclareMathOperator{\acl}{acl}
\DeclareMathOperator{\dcl}{dcl}
\DeclareMathOperator{\ch}{char}
\DeclareMathOperator{\ZFC}{ZFC}
\DeclareMathOperator{\Bew}{Bew}
\DeclareMathOperator{\CON}{CON}
%To get the number and the date
\newcounter{shnumber}
\newcommand{\setabgabe}[1]{\newcommand{\abgabe}{#1}}

%This is some stuff to have a pretty sheet, I have to dig into it
\pagestyle{empty}
\usetikzlibrary{arrows}

\newbox\rechtsbox
\newdimen\rechtsskip

\newcounter{groesse}
\newcommand{\gross}{%
\advance\topmargin by -5mm%
\advance\textheight by 10mm%
\advance\oddsidemargin by -3mm%
\advance\textwidth by 6mm%
\stepcounter{groesse}
}

\newcommand{\bw}{\vskip6ex minus 4ex\vfill\emph{(Bitte wenden.)}\pagebreak}

\gross\gross\gross\gross

\parindent0cm
\parskip.5ex

\renewcommand{\labelenumi}{\alph{enumi})}
\renewcommand{\labelenumii}{\roman{enumii})}

\newbox\gnBoxA
\newdimen\gnCornerHgt
\setbox\gnBoxA=\hbox{$\ulcorner$}
\global\gnCornerHgt=\ht\gnBoxA
\newdimen\gnArgHgt


\let\merkenum=\enumerate
\def\enumerate{\merkenum\itemsep0mm\topsep0mm}
\let\merkitem=\itemize
\def\itemize{\merkitem\itemsep0mm\topsep0mm}

%this is to take care of the environnement for exercises
\newcounter{aufgabe}

\long\def\vorAufgabe#1{\def\meinVorAufgabe{#1}}

\vorAufgabe{}
\def\leer{}

\def\softsymb{$^{\text{\manrotatedquadrifolium}}$}
\def\softfoot{}
\newcommand{\softfootinh}{\medskip{\footnotesize \softsymb\ Bei dieser Aufgabe gibt es kein richtig oder falsch;
Punkte gibt es, wenn sie bearbeitet wurde.\par}}

\newcommand{\voraufg}[1]{\vorAufgabe{#1}}

\newcommand{\aufg}[1][]{%
\vskip5ex minus 3ex
\stepcounter{aufgabe}
\ifx\meinVorAufgabe\leer\else
  \meinVorAufgabe\par\vskip1ex minus 0.5ex\vorAufgabe{}
\fi
\leavevmode\textbf{Aufgabe \arabic{aufgabe}#1.} }

\newcommand{\saufg}{\aufg[\softsymb]%
\let\softfoot=\softfootinh
}
\newcommand{\aaufg}[1][]{%
\vskip5ex minus 3ex
\stepcounter{aufgabe}
\ifx\meinVorAufgabe\leer\else
  \meinVorAufgabe\par\vskip1ex minus 0.5ex\vorAufgabe{}
\fi
\leavevmode\textbf{Anwesenheitsaufgabe \arabic{aufgabe}#1.} }

\newcommand{\naufg}[1][]{%
\vskip5ex minus 3ex
\stepcounter{aufgabe}
\ifx\meinVorAufgabe\leer\else
  \meinVorAufgabe\par\vskip1ex minus 0.5ex\vorAufgabe{}
\fi
\leavevmode\textbf{Nikolausaufgabe \arabic{aufgabe}#1.} }

\newcommand{\staraufg}[1][]{%
\vskip5ex minus 3ex
\stepcounter{aufgabe}
\ifx\meinVorAufgabe\leer\else
  \meinVorAufgabe\par\vskip1ex minus 0.5ex\vorAufgabe{}
\fi
\leavevmode\textbf{Aufgabe \arabic{aufgabe}#1\textsuperscript{*}\negphantom{\textsuperscript{*}}.} }
  
\newcommand{\pt}[1]{\hspace*{\fill}\textit{\hbox{#1 Punkte}}\par}
\newcommand{\pti}{\hspace*{\fill}\textit{1 Punkt}\par}
